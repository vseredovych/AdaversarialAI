% -*- mode: LaTeX; coding: utf8; -*-

\documentclass[a4paper,14pt]{extreport}

% правильне кодування
\usepackage[utf8]{inputenc}
\usepackage[T2A]{fontenc}
\usepackage[ukrainian]{babel}

% починати абзаци невеликим відступом першого рядка
\usepackage{indentfirst}
\usepackage[pdftex,unicode,bookmarks]{hyperref}

\usepackage{color}
\definecolor{bluegray}{RGB}{230,230,255}
\usepackage{listings}

\lstset{
	extendedchars=\true, % дозволити кирилицю в лінстингу
	inputencoding=utf8,
	breaklines=true,
	basicstyle=\ttfamily,
	numbers=left,
	frame=single,
	backgroundcolor=\color{bluegray}
}

% якщо прийдеться вставляти код (простіше ніж listings, але немає breaklines)
\usepackage{verbatim}

% інтервал - півтора. 
\usepackage{setspace}
\onehalfspacing

% поля
\usepackage{geometry}
\geometry{a4paper}
\geometry{left=35mm,right=15mm,top=20mm,bottom=20mm}
\geometry{headheight=2ex,headsep=10mm,footskip=10mm}

%для картинок
\usepackage{graphicx}
\usepackage{caption}


\begin{document}
    \begin{titlepage}%
       \begin{center}
       	{ЛЬВІВСЬКИЙ НАЦІОНАЛЬНИЙ УНІВЕРСИТЕТ \\ІМЕНІ ІВАНА ФРАНКА}\par
       	{ФAКУЛЬТЕТ ПРИКЛАДНОЇ МАТЕМАТИКИ ТА ІНФОРМАТИКИ\\ КАФЕДРА ОБЧИСЛЮВАЛЬНОЇ МАТЕМАТИКИ}\par
       	\begin{center}

       	\end{center}
       	\vspace{10mm}
       	\b{КУРСОВА РОБОТА}\par
       	{\small{на тему:}}\par
       	\vspace{20mm}
       	{\LARGE{\bf{\scshape{Аналіз атак на лінійні моделі машинного навчання}}}}\par
       	\vspace{5mm}
       	{}\par %subtitle
       \end{center}
   	   \vfill
   	   \hfill
   	   \begin{flushright}
   	   	\begin{minipage}[t]{80mm}
   	   		\flushright
   	   		студента III курсу\\
   	   		групи ПМп-31\\
   	   		{Середовича Віктора}\par
   	   		\vspace{2ex}
   	   		Науковий керівник:\\
   	   		{доцент Ю.А.Музичук}\\
   	   	\end{minipage}
   	   \end{flushright}
   	   \vspace{10mm}
   	   \begin{flushleft}
   	   	\begin{minipage}[t]{80mm}
   	   		\flushleft
   	   		Завідуючий кафедри обчислювальної математики\\
   	   		проф.  
   	   	\end{minipage}
   	   \end{flushleft}
   	   \vfill
   	   \vspace{10mm}
   	   \begin{center}Львів --- 2020\end{center}
   	   \stepcounter{page}
    \end{titlepage}




	\tableofcontents
	\newpage
	
	\chapter{Вступ} 
	Тема цієї роботи дуже важлива і потрібна. В ній розглядаються речі, надзвичайно
	важливі для подальшого розвитку науки, народного господарства та всього людства.

	\section{Постановка задачі} 
	Метою даної роботи є написання наукової роботи, і отримання за це оцінки, і як
	наслідок - диплому.
	
	\section{Основні поняття}
	Наукова робота - робота що полягає у написанні таких от розумних текстів, чи власне самі тексти.
	
	\chapter{Привіт, Світe!} 
	Основна частина даної роботи полягала у написанні програми. Нижче наводимо основний алгоритм її роботи, на мові C:
	
	\begin{lstlisting}
	#include &lt;stdio.h&gt;
	int main() 
	{ 
	printf("Hello, world!\n"); 
	return 0; 
	} 
	\end{lstlisting}


	\chapter{Альтернативні рішення} 
	Деякі дослідники пишуть свої роботи в програмах типу Microsoft Word. Але то не є труйово\cite{howto}.
	
	\chapter{Висновок} 
	Дана робота містить значний мій вклад, і перевершує попередні досягнення в багатьох напрямках. Окрім того, даний напрямок досліджень має значні перспективи
	подальшого розвитку. (Особливо добре було б, якби хтось вирішив проблему кирилиці в listings).
	
	\newpage
	\addcontentsline{toc}{chapter}{Література}
	\begin{thebibliography}{9}
		
		\bibitem{howto} Вікіпідручник \emph{Як написати курсову?}
		(\url{http://uk.wikibooks.org/wiki/%D0%AF%D0%BA_%D0%B2%D1%87%D0%B8%D1%82%D0%B8%D1%81%D1%8C_%D0%BA%D1%80%D0%B0%D1%89%D0%B5%3F/%D0%9A%D1%83%D1%80%D1%81%D0%BE%D0%B2%D1%96})
			
	\end{thebibliography}
		
\end{document}