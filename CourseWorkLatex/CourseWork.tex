% -*- mode: LaTeX; coding: utf8; -*-

\documentclass[a4paper,14pt]{extreport}

% правильне кодування
\usepackage[utf8]{inputenc}
\usepackage[T2A]{fontenc}
\usepackage[ukrainian]{babel}

% починати абзаци невеликим відступом першого рядка
\usepackage{indentfirst}
\usepackage[pdftex,unicode,bookmarks]{hyperref}

\usepackage{color}
\definecolor{bluegray}{RGB}{230,230,255}
\usepackage{listings}

\lstset{
	extendedchars=\true, % дозволити кирилицю в лінстингу
	inputencoding=utf8,
	breaklines=true,
	basicstyle=\ttfamily,
	numbers=left,
	frame=single,
	backgroundcolor=\color{bluegray}
}

% якщо прийдеться вставляти код (простіше ніж listings, але немає breaklines)
\usepackage{verbatim}

% інтервал - півтора. 
\usepackage{setspace}
\onehalfspacing

% поля
\usepackage{geometry}
\geometry{a4paper}
\geometry{left=35mm,right=15mm,top=20mm,bottom=20mm}
\geometry{headheight=2ex,headsep=10mm,footskip=10mm}

%
\usepackage{mathtools}
\usepackage{amsmath}

%для картинок
\usepackage{graphicx}
\usepackage{caption}

% lstlisting settings
\usepackage{xcolor}

% для часткових похыдних
\usepackage{physics}


% для алгоритмів
\usepackage[ruled,vlined]{algorithm2e}


\definecolor{codegreen}{rgb}{0,0.6,0}
\definecolor{codegray}{rgb}{0.5,0.5,0.5}
\definecolor{codepurple}{rgb}{0.58,0,0.82}
\definecolor{backcolour}{rgb}{0.95,0.95,0.92}

\lstdefinestyle{mystyle}{
	backgroundcolor=\color{backcolour},   
	commentstyle=\color{codegreen},
	keywordstyle=\color{magenta},
	numberstyle=\tiny\color{codegray},
	stringstyle=\color{codepurple},
	basicstyle=\ttfamily\footnotesize,
	breakatwhitespace=false,         
	breaklines=true,                 
	captionpos=b,                    
	keepspaces=true,                 
	numbers=left,                    
	numbersep=5pt,                  
	showspaces=false,                
	showstringspaces=false,
	showtabs=false,                  
	tabsize=2
}
\lstset{style=mystyle}


% custom commands
\newcommand{\tran}{^{T}}
\newcommand{\ith}{^{(i)}}
%\newcommand{\someth}[1]{^{(#1)}}


\begin{document}
	% ======================================================================================== %
	\begin{titlepage}%
    	\begin{center}
	    	{ЛЬВІВСЬКИЙ НАЦІОНАЛЬНИЙ УНІВЕРСИТЕТ \\ІМЕНІ ІВАНА ФРАНКА}\par
	       	{ФAКУЛЬТЕТ ПРИКЛАДНОЇ МАТЕМАТИКИ ТА ІНФОРМАТИКИ\\ КАФЕДРА ОБЧИСЛЮВАЛЬНОЇ МАТЕМАТИКИ}\par
			\begin{center}
	
	        \end{center}
	        \vspace{10mm}
	        \b{КУРСОВА РОБОТА}\par
	        {\small{на тему:}}\par
	        \vspace{20mm}
	        {\LARGE{\bf{\scshape{Аналіз атак на лінійні моделі машинного навчання}}}}\par
	        \vspace{5mm}
	        {}\par %subtitle
        \end{center}
	   	
	   	\vfill
	   	\hfill
		\begin{flushright}
   	   		\begin{minipage}[t]{80mm}
   	   			\flushright
	   	   		студента III курсу\\
	   	   		групи ПМп-31\\
	   	   		{Середовича Віктора}\par
	   	   		\vspace{2ex}
	   	   		Науковий керівник:\\
	   	   		{доцент Ю.А.Музичук}\\
   	   		\end{minipage}
   	   \end{flushright}
      
   	   \vspace{10mm}
   	   \begin{flushleft}
   	   \begin{minipage}[t]{80mm}
	   	   \flushleft
	   	   Завідуючий кафедри обчислювальної математики\\
	   	   проф.  
   	   \end{minipage}
   	   \end{flushleft}
   	   \vfill
   	   \vspace{10mm}
   	   
   	   \begin{center}Львів --- 2020\end{center}
   	   \stepcounter{page}
    \end{titlepage}
	% ======================================================================================== %
	
	
	% ======================================================================================== %
	\tableofcontents
	\newpage
	% ======================================================================================== %
	
	
	% ======================================================================================== %
	% ============================================ %
	\chapter{Вступ}
	\textgreater\textbf{TODO}
	Машинне начання та штучний інтелект активно використовується у різних областях нашого життя, та допомагає у вирішенні таких задач як розпізнавання дорожніх знаків, облич, визначення ризику захворювання та багато іншого.
	А з поширенням його використання, також збільшуєтья і ризик нападів зловмисників на ці алгоритми, що може привести, до трагічних наслідків. Тому варто досліджувати тему нападів на алгоритми машинного навчання, та знати як захистити свою модель. \par
	В межах теми цієї роботи будуть розглядатись різні атаки на лінійні моделі машинного начання, та методи їх захисту.
	% ============================================ %
	
	
	% ============================================ %
	\section{Постановка задачі} 
	\textit{Мета} даної роботи полягає у тому, щоб дослідити ефективність атак на лінійні моделі машинного навчання, та визначити методи захисту від них. \par
	Виходячи з мети, визначені завдання роботи:
	\begin{itemize}  
	\item Практична реалізація та дослідження методів атак
	\item Визначення методів захисту
	\end{itemize}
	% ============================================ %
	% ======================================================================================== %
	
	% ======================================================================================== %
	% ============================================ %
	\chapter{Тренування моделі}
	В якості прикладу лінійного методу машинного навчання, на який будуть сдійснюватись атаки, буде використовуватись модифікований алгоритм логістичної регресії для мультикласової класифікації.
	
	\section{Модифікована логістична регресія}
	Задача прикладу $x \in R^{n_x}$, знайти $\hat{y}=P(y = 1 \mid x), 0 \leq \hat{y} \leq 1$
	Шукатимемо у вигляді $\hat{y} = \sigma (\omega\tran x + b)$, де
	\begin{itemize}
		\item параметри $\omega \in R^{n_x}, b \in R$ - невідомі, потрібно знайти оптимальні для для даної задачі
		\item $\sigma(z) = \frac{1}{1+ e^{-z}}$ - сигмоїд
	\end{itemize}
	Для кожного прикладу з тренувального датасету потрібно обчислити:
	\begin{itemize}
		\item $ z\ith = \omega\tran x\ith + b $
		\item $ y\ith = \sigma\ith (z\ith) $
	\end{itemize}
	де $\sigma(z\ith) = \frac{1}{1 + e^{-z\ith}}$, так щоб $\hat{y}\ith \approx y\ith $ \newline
	Для кожного прикладу визначена функція втрати:
	$$L(\hat{y}\ith, y\ith) = -y\ith \log(\hat{y}\ith) - (1 - y\ith) \log(1 - \hat{y}\ith)$$
	На всіх прикладах обчислюємо штрафну функцію як:
	$$J(w, b) = \frac{1}{m} \sum_{i=1}^{m} L(\hat{y}\ith, y\ith)$$
	
	Задача полягає в тому щоб знайти параметри $w \in R^n_x, b\in R$ що мінімізують функцію $J(\omega, b)$
	Для цього будемо використовувати градієнтний спуск 
	
	
	
	% ============================================ %
	% ======================================================================================== %


	% ======================================================================================== %
	\chapter{Класифікація атак} 
	% ============================================ %
	\section{Метод Швидкого Градієнту} 
	Першим методом атаки який буде розглядатись є методу швидкого градієнту (Fast Gradient Sign Method).
	Ідея полягає в тому ..=.
	
	\begin{align*}
		z\ith = \omega\tran x\ith + b \quad
		\hat{y}\ith = a\ith = \sigma(z\ith) \quad
		\sigma(z\ith) = \frac{1}{1 + e^{-z\ith}} \quad
	\end{align*}
	\begin{align*}
		L(\hat{y}\ith, y\ith) = -y\ith \log(\hat{y}\ith) - (1 - y\ith) \log(1 - \hat{y}\ith)
	\end{align*}
	\begin{align*}
		da\ith = \pdv{ L(a\ith, y\ith)}{a\ith} = -y\ith \cdot \frac{1}{a\ith} -(1 - y\ith) \cdot \frac{1}{1-a\ith} = -\frac{y\ith}{a\ith} + \frac{1-y\ith}{1-a\ith}
	\end{align*}
	\begin{align*}
		dz\ith &= \pdv{L(a\ith, y\ith)}{a\ith} \cdot \pdv{a\ith}{z\ith} =
		-\frac{y\ith a\ith (1-a\ith)}{a\ith} + \frac{(1-y\ith) a\ith (1-a\ith)}{1-a\ith} = \\
		& = -y\ith + a\ith y\ith + a\ith - y\ith a\ith  = a\ith - y\ith
	\end{align*}
	%\begin{align*}
	%	dz\ith &= \pdv{L(a\ith, y\ith)}{w\ith_j} \cdot \pdv{z\ith}{w\ith_j} = x\ith_j dz\ith
	%\end{align*}
	%\begin{align*}
	%	db\ith &= \pdv{L(a\ith, y\ith)}{b\ith_j} \cdot \pdv{z\ith}{b\ith_j} = dz\ith
	%\end{align*}
	\begin{align*}
		dx\ith_j &= \pdv{L(a\ith, y\ith)}{x\ith_j} \cdot \pdv{z\ith}{x\ith_j} = 
		(w\tran)\ith_j (a\ith - y\ith) = (w\tran)\ith_j dz
	\end{align*}



	% ============================================ %
	
	% ============================================ %
	\section{Нецілеспрямовані атаки} 
	Основна частина даної роботи полягала у написанні програми. Нижче наводимо основний алгоритм її роботи, на мові C:
	% ============================================ %
	
	% ============================================ %
	\begin{lstlisting}[language=Python]
	% ============================================ %
	#include &lt;stdio.h&gt;
	int main() 
	{ 
	printf("Hello, world!\n"); 
	return 0; 
	} 
	\end{lstlisting}[language=Python]

	\lstset{language=Python}
	\begin{lstlisting}
	def fit(self, X, w, b, y, alpha, max_iters, predict_func):
		# Check that X and y have correct shape
		self.w = w
		self.b = b
		
		self.y_ = np.expand_dims(y.T, axis=1)
		self.X_ = X.T
		
		self.num_iters = 0
		self.X_ = self._gradient_descent(self.X_, self.y_, self.w, self.b, alpha, max_iters, predict_func)
		
		def _cost_function(self, X, Y, A):
		m = X.shape[0]
		if m == 0:
		return None
		
		J = (1 / m) * np.sum(-Y * np.log(A) - (1 - Y) * np.log(1 - A))
		return J
	\end{lstlisting}
	
	\begin{algorithm}[H]
		\SetAlgoLined
		\KwResult{Write here the result }
		initialization\;
		\While{While condition}{
			instructions\;
			\eIf{condition}{
				instructions1\;
				instructions2\;
			}{
				instructions3\;
			}
		}
		\caption{How to write algorithms}
	\end{algorithm}
	
	% ============================================ %
	
	
	% ============================================ %
	\chapter{Альтернативні рішення} 
	Деякі дослідники пишуть свої роботи в програмах типу Microsoft Word. Але то не є труйово\cite{howto}.
	% ============================================ %
	
	
	% ============================================ %	
	\chapter{Висновок} 
	Дана робота містить значний мій вклад, і перевершує попередні досягнення в багатьох напрямках. Окрім того, даний напрямок досліджень має значні перспективи
	подальшого розвитку. (Особливо добре було б, якби хтось вирішив проблему кирилиці в listings).
	% ======================================================================================== %
	
	
	% ============================================ %	
	\newpage
	\addcontentsline{toc}{chapter}{Література}
	\begin{thebibliography}{9}
		
		\bibitem{howto} Вікіпідручник \emph{Як написати курсову?}
		(\url{http://uk.wikibooks.org/wiki/%D0%AF%D0%BA_%D0%B2%D1%87%D0%B8%D1%82%D0%B8%D1%81%D1%8C_%D0%BA%D1%80%D0%B0%D1%89%D0%B5%3F/%D0%9A%D1%83%D1%80%D1%81%D0%BE%D0%B2%D1%96})
			
		\end{thebibliography}
	% ============================================ %
	
\end{document}